

\documentclass[letterpaper,11pt]{article}


\usepackage{latexsym}
\usepackage[empty]{fullpage}
\usepackage{titlesec}
\usepackage{marvosym}
\usepackage[usenames,dvipsnames]{color}
\usepackage{verbatim}
\usepackage{enumitem}
\usepackage[hidelinks]{hyperref}
\usepackage{fancyhdr}

\usepackage[utf8]{inputenc}
\usepackage[russian]{babel}
%\usepackage[english]{babel}
\usepackage{tabularx}
\input{glyphtounicode}


%----------FONT OPTIONS----------
%
%\renewcommand\familydefault{\sfdefault} 
%\usepackage[T1]{fontenc}

% sans-serif
%\usepackage[sfdefault]{FiraSans}
%\usepackage[sfdefault]{roboto}
%\usepackage[sfdefault]{noto-sans}
%\usepackage[default]{sourcesanspro}

% serif
% \usepackage{charter}
% \usepackage{CormorantGaramond}


\pagestyle{fancy}
\fancyhf{} % clear all header and footer fields
\fancyfoot{}
\renewcommand{\headrulewidth}{0pt}
\renewcommand{\footrulewidth}{0pt}

% Adjust margins
\addtolength{\oddsidemargin}{-0.5in}
\addtolength{\evensidemargin}{-0.5in}
\addtolength{\textwidth}{1in}
\addtolength{\topmargin}{-.5in}
\addtolength{\textheight}{1.0in}

\urlstyle{same}

\raggedbottom
\raggedright
\setlength{\tabcolsep}{0in}

% Sections formatting
\titleformat{\section}{
  \vspace{-4pt}\scshape\raggedright\large
}{}{0em}{}[\color{black}\titlerule \vspace{-5pt}]

% Ensure that generate pdf is machine readable/ATS parsable
\pdfgentounicode=1

%-------------------------
% Custom commands
\newcommand{\resumeItem}[1]{
  \item\small{
    {#1 \vspace{-2pt}}
  }
}

\newcommand{\resumeSubheading}[4]{
  \vspace{-2pt}\item
    \begin{tabular*}{0.97\textwidth}[t]{l@{\extracolsep{\fill}}r}
      \textbf{#1} & #2 \\
      \textit{\small#3} & \textit{\small #4} \\
    \end{tabular*}\vspace{-7pt}
}
% \newcommand{\resumeSubheading}[4]{
%   \vspace{-2pt}\item
%     \begin{tabular*}{0.97\textwidth}[t]{l@{\extracolsep{\fill}}p{6cm}}
%       \textbf{#1} & \raggedleft #2 \\
%       \textit{\small#3} & \raggedleft \textit{\small #4} \\
%     \end{tabular*}\vspace{-7pt}
% }

\newcommand{\resumeSubSubheading}[2]{
    \item
    \begin{tabular*}{0.97\textwidth}{l@{\extracolsep{\fill}}r}
      \textit{\small#1} & \textit{\small #2} \\
    \end{tabular*}\vspace{-7pt}
}

\newcommand{\resumeProjectHeading}[2]{
    \item
    \begin{tabular*}{0.97\textwidth}{l@{\extracolsep{\fill}}r}
      \small#1 & #2 \\
    \end{tabular*}\vspace{-7pt}
}

\newcommand{\resumeSubItem}[1]{\resumeItem{#1}\vspace{-4pt}}

\renewcommand\labelitemii{$\vcenter{\hbox{\tiny$\bullet$}}$}

\newcommand{\resumeSubHeadingListStart}{\begin{itemize}[leftmargin=0.15in, label={}]}
\newcommand{\resumeSubHeadingListEnd}{\end{itemize}}
\newcommand{\resumeItemListStart}{\begin{itemize}}
\newcommand{\resumeItemListEnd}{\end{itemize}\vspace{-5pt}}

%-------------------------------------------
%%%%%%  RESUME STARTS HERE  %%%%%%%%%%%%%%%%%%%%%%%%%%%%


\begin{document}

%----------HEADING----------
% \begin{tabular*}{\textwidth}{l@{\extracolsep{\fill}}r}
%   \textbf{\href{http://sourabhbajaj.com/}{\Large Sourabh Bajaj}} & Email : \href{mailto:sourabh@sourabhbajaj.com}{sourabh@sourabhbajaj.com}\\
%   \href{http://sourabhbajaj.com/}{http://www.sourabhbajaj.com} & Mobile : +1-123-456-7890 \\
% \end{tabular*}

\begin{center}
    \textbf{\Huge \scshape Щербак Артём} \\ \vspace{10pt}
    \text{\fontsize{18}{24} \scshape ML Engineer} \\ \vspace{10pt}
    \small +7(962)-280-27-37 $|$ 
    \href{mailto:shcherbak.artyom@yandex.ru}{\underline{shcherbak.artyom@yandex.ru}} $|$ 
    %\href{https://linkedin.com/in/...}{\underline{linkedin.com/in/jake}} $|$
    \href{https://github.com/RtemShcherbak}{\underline{github.com/RtemShcherbak}} $|$
    \href{https://t.me/werevvindel}{\underline{t.me/werevvindel}}
\end{center}

%----------- ABOUT -----------
\section{О себе}
  \resumeSubHeadingListStart
  {ML-инженер с практическим опытом в разработке и оптимизации моделей для анализа видео и video-prediction'a и работы с геопространственными данными. Имею опыт построения моделей от препроцессинга и настройки инфраструктуры до внедрения api в сервис. Заинтересован в глубоком понимании моделей и интерпретируемости решений.}
  \resumeSubHeadingListEnd

%-----------PROGRAMMING SKILLS-----------
\section{Навыки}
 \begin{itemize}[leftmargin=0.15in, label={}]
    \small{\item{
     \textbf{Languages}{: Python, C++, SQL, HTML/CSS} \\
     \textbf{Frameworks}{: Flask, PyTorch, PyTorch Lightning} \\
     \textbf{Developer Tools}{: Git, Docker, VS Code, PyCharm, Google Colab} \\
     \textbf{Libraries}{: Pandas, NumPy, Matplotlib, Seaborn, Scikit-learn, CatBoost, Dask, PyTorch, PyTorch Lightning, Transformers, OpenCV, OpenSTL, Xarray, Triton}
    }}
 \end{itemize}

%-----------EXPERIENCE-----------
\section{Опыт работы}
  \resumeSubHeadingListStart

    \resumeSubheading
      {Специалист по машинному обучению }{03.03.2024 -- н.в.}
      {Моринтех $|$ Стек: Pandas, OpenCV, OpenSTL, PyTorch, PyTorch Lightning, Xarray, Git, Docker }{}
      \resumeItemListStart
        \resumeItem{ Проводил анализ качества геопространственных данных и их визуализацию. }
        % \resumeItem{ Подготавливал данные для обучения моделей на их основе. }
        \resumeItem{ Разработал и внедрил кастомные архитектуры для прогноза видео (CTM, SimVP, PredFormer) с использованием PyTorch Lightning и кастомных Callback’ов, что позволило повысить стабильность и масштабируемость экспериментов в условиях ограниченных ресурсов GPU. }
        \resumeItem{Оптимизировал обучение моделей с использованием DDP и FSDP, переписал систему логирования и обработки Checkpoint’ов, что снизило время экспериментов на 30\% и улучшило воспроизводимость в распределённой среде.}
        \resumeItem{Адаптировал систему хуков EMA и анализа активаций под PyTorch Lightning, переписав механизм приоритетов и вызова, в результате чего стало возможным гибко управлять логикой обучения.}
        \resumeItem{Реализовал sparse attention-механизмы, что сократило потребление памяти и ускорило обработку длинных последовательностей.}
        \resumeItem{Создал гибкий токенизатор видео и модуль позиционного кодирования, поддерживающий пространственно-ориентированный sparse attention, что улучшило качество генерации предсказанных кадров на сложных датасетах.}
        \resumeItem{Подготовил техническое описание программного продукта и участвовал в составлении итогового отчёта на этапе УГТ-8, что обеспечило соответствие документации требованиям заказчика и способствовало успешной защите этапа.}
      \resumeItemListEnd


    \resumeSubheading
      {Инженер}{20.03.2021 -- 23.02.2024}
      {Роскосмос $|$ Стек: Python, NumPy, Sklearn, CatBoost, SpaCy, SciPy, Pandas, Flask, Linux}{}
      \resumeItemListStart
        \resumeItem{Разработал программное обеспечение для анализа измерений геометрических характеристик антенных систем (выполнялось для задач импортозамещения).}
        \resumeItem{ Реализовал систему автоматической обработки результатов численных экспериментов. Это позволило  сократить время составления и оформления отчётов на 50\%.}
        \resumeItem{ Разработал и обучил модель для проведения оценки соответствия кандидата заданной вакансии, что позволило уменьшить время, затрачиваемое на рассмотрение неподходящих кандидатов. }
        \resumeItem{ Разработал и обучил модель для прогноза зарплатной вилки кандидата по его резюме, что позволило сократить время оценивания кандидата в 1,5 раза. }
        \resumeItem{Составлял коммерческие запросы и предложения для предприятий-подрядчиков.}
      \resumeItemListEnd
  \resumeSubHeadingListEnd


% %-----------PROJECTS-----------
% \section{Проекты}
%     \resumeSubHeadingListStart
%       \resumeProjectHeading
%           {\textbf{Веб-сервис QA} $|$ Стек: \emph{Python, PyTorch Lightning, Flask, Docker, Linux}}{25.10.2023 -- н.в.}
%           \resumeItemListStart
%             \resumeItem{Разработал веб-приложение с использованием Flask и шаблонов Bootstrap в качестве интерфейса}
%             \resumeItem{Дообучил модель Т5 на русскоязычном датасете для задач QA}
%             \resumeItem{Арендовал и настроил удалённый виртуальный сервер}
%             \resumeItem{Упаковал модель и бэкенд сайта в докер контейнеры, объединил контейнеры в сеть и развернул на сервере при помощи docker-compose файла}
%           \resumeItemListEnd

%     \resumeSubHeadingListEnd

%-----------EDUCATION-----------
\section{Образование}
  \resumeSubHeadingListStart
    \resumeSubheading
        {МФТИ}{01.09.2022 -- 30.06.2023}
        {Data Scientist}{}
        
    \resumeSubheading
        {МГТУ им. Н. Э. Баумана}{05.08.2017 -- 30.06.2023}
        {Инженер-контруктор ракетной техники}{}
  \resumeSubHeadingListEnd

%

%-------------------------------------------
\end{document}
